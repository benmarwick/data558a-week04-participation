\documentclass[]{article}
\usepackage{lmodern}
\usepackage{amssymb,amsmath}
\usepackage{ifxetex,ifluatex}
\usepackage{fixltx2e} % provides \textsubscript
\ifnum 0\ifxetex 1\fi\ifluatex 1\fi=0 % if pdftex
  \usepackage[T1]{fontenc}
  \usepackage[utf8]{inputenc}
\else % if luatex or xelatex
  \ifxetex
    \usepackage{mathspec}
  \else
    \usepackage{fontspec}
  \fi
  \defaultfontfeatures{Ligatures=TeX,Scale=MatchLowercase}
\fi
% use upquote if available, for straight quotes in verbatim environments
\IfFileExists{upquote.sty}{\usepackage{upquote}}{}
% use microtype if available
\IfFileExists{microtype.sty}{%
\usepackage[]{microtype}
\UseMicrotypeSet[protrusion]{basicmath} % disable protrusion for tt fonts
}{}
\PassOptionsToPackage{hyphens}{url} % url is loaded by hyperref
\usepackage[unicode=true]{hyperref}
\hypersetup{
            pdftitle={data558a\_week04\_participation},
            pdfauthor={Will Wright},
            pdfborder={0 0 0},
            breaklinks=true}
\urlstyle{same}  % don't use monospace font for urls
\usepackage[margin=1in]{geometry}
\usepackage{color}
\usepackage{fancyvrb}
\newcommand{\VerbBar}{|}
\newcommand{\VERB}{\Verb[commandchars=\\\{\}]}
\DefineVerbatimEnvironment{Highlighting}{Verbatim}{commandchars=\\\{\}}
% Add ',fontsize=\small' for more characters per line
\usepackage{framed}
\definecolor{shadecolor}{RGB}{248,248,248}
\newenvironment{Shaded}{\begin{snugshade}}{\end{snugshade}}
\newcommand{\KeywordTok}[1]{\textcolor[rgb]{0.13,0.29,0.53}{\textbf{#1}}}
\newcommand{\DataTypeTok}[1]{\textcolor[rgb]{0.13,0.29,0.53}{#1}}
\newcommand{\DecValTok}[1]{\textcolor[rgb]{0.00,0.00,0.81}{#1}}
\newcommand{\BaseNTok}[1]{\textcolor[rgb]{0.00,0.00,0.81}{#1}}
\newcommand{\FloatTok}[1]{\textcolor[rgb]{0.00,0.00,0.81}{#1}}
\newcommand{\ConstantTok}[1]{\textcolor[rgb]{0.00,0.00,0.00}{#1}}
\newcommand{\CharTok}[1]{\textcolor[rgb]{0.31,0.60,0.02}{#1}}
\newcommand{\SpecialCharTok}[1]{\textcolor[rgb]{0.00,0.00,0.00}{#1}}
\newcommand{\StringTok}[1]{\textcolor[rgb]{0.31,0.60,0.02}{#1}}
\newcommand{\VerbatimStringTok}[1]{\textcolor[rgb]{0.31,0.60,0.02}{#1}}
\newcommand{\SpecialStringTok}[1]{\textcolor[rgb]{0.31,0.60,0.02}{#1}}
\newcommand{\ImportTok}[1]{#1}
\newcommand{\CommentTok}[1]{\textcolor[rgb]{0.56,0.35,0.01}{\textit{#1}}}
\newcommand{\DocumentationTok}[1]{\textcolor[rgb]{0.56,0.35,0.01}{\textbf{\textit{#1}}}}
\newcommand{\AnnotationTok}[1]{\textcolor[rgb]{0.56,0.35,0.01}{\textbf{\textit{#1}}}}
\newcommand{\CommentVarTok}[1]{\textcolor[rgb]{0.56,0.35,0.01}{\textbf{\textit{#1}}}}
\newcommand{\OtherTok}[1]{\textcolor[rgb]{0.56,0.35,0.01}{#1}}
\newcommand{\FunctionTok}[1]{\textcolor[rgb]{0.00,0.00,0.00}{#1}}
\newcommand{\VariableTok}[1]{\textcolor[rgb]{0.00,0.00,0.00}{#1}}
\newcommand{\ControlFlowTok}[1]{\textcolor[rgb]{0.13,0.29,0.53}{\textbf{#1}}}
\newcommand{\OperatorTok}[1]{\textcolor[rgb]{0.81,0.36,0.00}{\textbf{#1}}}
\newcommand{\BuiltInTok}[1]{#1}
\newcommand{\ExtensionTok}[1]{#1}
\newcommand{\PreprocessorTok}[1]{\textcolor[rgb]{0.56,0.35,0.01}{\textit{#1}}}
\newcommand{\AttributeTok}[1]{\textcolor[rgb]{0.77,0.63,0.00}{#1}}
\newcommand{\RegionMarkerTok}[1]{#1}
\newcommand{\InformationTok}[1]{\textcolor[rgb]{0.56,0.35,0.01}{\textbf{\textit{#1}}}}
\newcommand{\WarningTok}[1]{\textcolor[rgb]{0.56,0.35,0.01}{\textbf{\textit{#1}}}}
\newcommand{\AlertTok}[1]{\textcolor[rgb]{0.94,0.16,0.16}{#1}}
\newcommand{\ErrorTok}[1]{\textcolor[rgb]{0.64,0.00,0.00}{\textbf{#1}}}
\newcommand{\NormalTok}[1]{#1}
\usepackage{graphicx,grffile}
\makeatletter
\def\maxwidth{\ifdim\Gin@nat@width>\linewidth\linewidth\else\Gin@nat@width\fi}
\def\maxheight{\ifdim\Gin@nat@height>\textheight\textheight\else\Gin@nat@height\fi}
\makeatother
% Scale images if necessary, so that they will not overflow the page
% margins by default, and it is still possible to overwrite the defaults
% using explicit options in \includegraphics[width, height, ...]{}
\setkeys{Gin}{width=\maxwidth,height=\maxheight,keepaspectratio}
\IfFileExists{parskip.sty}{%
\usepackage{parskip}
}{% else
\setlength{\parindent}{0pt}
\setlength{\parskip}{6pt plus 2pt minus 1pt}
}
\setlength{\emergencystretch}{3em}  % prevent overfull lines
\providecommand{\tightlist}{%
  \setlength{\itemsep}{0pt}\setlength{\parskip}{0pt}}
\setcounter{secnumdepth}{0}
% Redefines (sub)paragraphs to behave more like sections
\ifx\paragraph\undefined\else
\let\oldparagraph\paragraph
\renewcommand{\paragraph}[1]{\oldparagraph{#1}\mbox{}}
\fi
\ifx\subparagraph\undefined\else
\let\oldsubparagraph\subparagraph
\renewcommand{\subparagraph}[1]{\oldsubparagraph{#1}\mbox{}}
\fi

% set default figure placement to htbp
\makeatletter
\def\fps@figure{htbp}
\makeatother

\usepackage{booktabs}
\usepackage{longtable}
\usepackage{array}
\usepackage{multirow}
\usepackage{wrapfig}
\usepackage{float}
\usepackage{colortbl}
\usepackage{pdflscape}
\usepackage{tabu}
\usepackage{threeparttable}
\usepackage{threeparttablex}
\usepackage[normalem]{ulem}
\usepackage{makecell}
\usepackage{xcolor}

\title{data558a\_week04\_participation}
\author{Will Wright}
\date{02 February 2020}

\begin{document}
\maketitle

\begin{Shaded}
\begin{Highlighting}[]
\KeywordTok{cat}\NormalTok{(}\StringTok{"Example of trivial r code in a chunk"}\NormalTok{)}
\end{Highlighting}
\end{Shaded}

\begin{verbatim}
## Example of trivial r code in a chunk
\end{verbatim}

\begin{Shaded}
\begin{Highlighting}[]
\NormalTok{x <-}\StringTok{ }\DecValTok{2} \OperatorTok{+}\StringTok{ }\DecValTok{2}
\end{Highlighting}
\end{Shaded}

Example of in-line R code, which references that chunk: x = 4.

\begin{Shaded}
\begin{Highlighting}[]
\NormalTok{g <-}\StringTok{ }\KeywordTok{ggplot}\NormalTok{(starwars[}\KeywordTok{which}\NormalTok{(starwars}\OperatorTok{$}\NormalTok{species}\OperatorTok{==}\StringTok{"Human"} \OperatorTok{&}\StringTok{ }\KeywordTok{is.na}\NormalTok{(starwars}\OperatorTok{$}\NormalTok{height)}\OperatorTok{==}\OtherTok{FALSE}\NormalTok{),], }\KeywordTok{aes}\NormalTok{(}\DataTypeTok{x =} \KeywordTok{reorder}\NormalTok{(}\KeywordTok{factor}\NormalTok{(name), }\OperatorTok{-}\NormalTok{height), }\DataTypeTok{y =}\NormalTok{ height))}
\NormalTok{g }\OperatorTok{+}\StringTok{ }\KeywordTok{geom_bar}\NormalTok{(}\DataTypeTok{stat =} \StringTok{"identity"}\NormalTok{) }\OperatorTok{+}
\StringTok{  }\KeywordTok{theme_bw}\NormalTok{() }\OperatorTok{+}
\StringTok{  }\KeywordTok{labs}\NormalTok{(}\DataTypeTok{title =} \StringTok{"Heights of Star Wars Characters"}\NormalTok{,}
       \DataTypeTok{x =} \StringTok{"Character"}\NormalTok{,}
       \DataTypeTok{y =} \StringTok{"Height"}\NormalTok{) }\OperatorTok{+}
\StringTok{  }\KeywordTok{theme}\NormalTok{(}\DataTypeTok{axis.text.x =} \KeywordTok{element_text}\NormalTok{(}\DataTypeTok{angle =} \DecValTok{40}\NormalTok{, }\DataTypeTok{hjust =} \DecValTok{1}\NormalTok{))}
\end{Highlighting}
\end{Shaded}

\begin{figure}
\centering
\includegraphics{data558a_week04_participation_files/figure-latex/starwars-plot-1.pdf}
\caption{A ggplot of human Starwars character heights}
\end{figure}

Some text referring to @ref(fig:starwars-plot)

\begin{Shaded}
\begin{Highlighting}[]
\KeywordTok{head}\NormalTok{(starwars[}\KeywordTok{which}\NormalTok{(starwars}\OperatorTok{$}\NormalTok{species}\OperatorTok{==}\StringTok{"Human"}\NormalTok{),],}\DecValTok{20}\NormalTok{) }\OperatorTok
\StringTok{  }\KeywordTok{select}\NormalTok{(name, height) }\OperatorTok
\StringTok{  }\KeywordTok{arrange}\NormalTok{(}\KeywordTok{desc}\NormalTok{(height)) }\OperatorTok
\StringTok{  }\NormalTok{knitr}\OperatorTok{::}\KeywordTok{kable}\NormalTok{(}\DataTypeTok{caption =} \StringTok{"A knitr kable table of starwars data"}\NormalTok{)}
\end{Highlighting}
\end{Shaded}

\begin{table}

\caption{\label{tab:starwars-tbl}A knitr kable table of starwars data}
\centering
\begin{tabular}[t]{l|r}
\hline
name & height\\
\hline
Darth Vader & 202\\
\hline
Qui-Gon Jinn & 193\\
\hline
Anakin Skywalker & 188\\
\hline
Biggs Darklighter & 183\\
\hline
Boba Fett & 183\\
\hline
Obi-Wan Kenobi & 182\\
\hline
Wilhuff Tarkin & 180\\
\hline
Han Solo & 180\\
\hline
Jek Tono Porkins & 180\\
\hline
Owen Lars & 178\\
\hline
Lando Calrissian & 177\\
\hline
Lobot & 175\\
\hline
Luke Skywalker & 172\\
\hline
Wedge Antilles & 170\\
\hline
Palpatine & 170\\
\hline
Finis Valorum & 170\\
\hline
Beru Whitesun lars & 165\\
\hline
Leia Organa & 150\\
\hline
Mon Mothma & 150\\
\hline
Arvel Crynyd & NA\\
\hline
\end{tabular}
\end{table}

We can see, in table @ref(tab:starwars-tbl) that people have heights.

And finally, here is an image of some characters' relative heights:\\
\includegraphics{star_wars_heights.png}

\end{document}
